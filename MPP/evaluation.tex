\section{Evaluation}
\subsection{Performance}
In order to understand how the implementation performs against alternatives, both as it is being iteratively improved and prior to release. Results of various performance tests shall be provided.

For each feature present, the framework will be compared against existing projects or language constructs that provide such functionality. When demonstrating a functional Map implementation, the time taken for a language's default Map implementation, alongside the results of my library performing on both the GPU and CPU as compute devices, shall be shown.

Such comparisons shall be enhanced by providing granulated details of the time taken for various stages of execution, in order to highlight which aspects scale well to GPU environments and which do not.

In addition to 'artificial' benchmarks that show how individual actions perform, several equivalent code samples shall be produced using competing methodologies. Evaluating these shall show if performing some task, such as machine-learning, can be accelerated when including the project's framework into an existing high-level implementation - without sacrificing clarity.
\subsection{Usability}
Since the project aims for usability of the outcome and not simply the highest performance possible, several user-evaluation trials shall be conducted in order to influence design decisions.

Real-world users shall be observed applying the framework's functions to personal projects or experimental scenarios. Aspects of the provided framework that they find issues with shall be recorded. It will be investigated whether anything can be improved in these cases.

The produced framework shall be packaged in a way that makes it simple to obtain and include into projects by interested parties.
\subsection{Portability}
In addition to ensuring that the project is simple to obtain and utilise, it shall be tested on many compatible build environments and hardware architectures as possible. This shall demonstrate its suitability as a general solution.
